\Chapter{Bevezetés}
A modern részvénytársaságok múltja egészen a 17. századig nyúlik vissza. Ekkor alapították az első részvénytársaságot Amszterdamban. Kezdetben csak egyetlen társasággal kereskedtek, az osztalékfizetés pedig néhány évvel később kezdődött el.
A tőzsde a vállalatok számára az egyik legfontosabb pénzszerzési módszer. Nyilvánosan kereskedhetnek, és további pénzügyi tőkét gyűjthetnek a terjeszkedéshez azáltal, hogy a vállalat tulajdonrészeit nyilvános piacon értékesítik. A történelem azt mutatja, hogy a részvények és más eszközök árfolyama fontos része a gazdasági tevékenység dinamikájának. Gyakran tekintik egy ország gazdasági erejének és fejlődése elsődleges mutatójának.

1971-ben megtörtént az áttörés, elkezdte a működését a Nasdaq, a világ első elektronikus tőzsdéje. Eleinte csak egy jegyzési rendszer volt, nem adott lehetőséget a kereskedésre. Mivel csökkentette a brókerek nyereségét azáltal, hogy az árkülönbséget csökkentette a kínált és a keresett összegek között, ezért nem kedvelték. Az évek során tovább bővült a Nasdaq, már automatikus kereskedési rendszereket is implementáltak.

Napjainkban egyre nagyobb embertömeget ér el a tőzsdézés. Az internet terjedése platformot kínált a részvények világának, ami nem is annyira veszélyes, mint elsőre gondolnánk. Több weboldal, könyv készült már, mely bemutatja miért és hogyan érdemes kereskedni. Már egy kattintással tulajdonosai lehetünk például a Tesla vállalatnak, és a világon bárhonnan nyomon követhetjük a részvények mozgását, s ez nagyban megkönnyíti a dolgunkat.

A kereskedésnek több stílusa létezik. Az egyik, amikor a számítógép előtt ülünk nyolc órát, és nézzük mikor csökken a részvény, hogy vásároljunk, böngésszük a fórumokat, hogy milyen vállalatokba fektetnek mások. Ez elég monoton és stresszes munka véleményem szerint, hiszen grafikonokat kellene nézegetni nap, mint nap, illetve izgulni, hogy ne lefelé menjen az árfolyam. Ennek apropóján érdemes elgondolkodni azon, hogy a tudást, amit könyvekből, weboldalakról, publikációkból összegyűjtünk, azt egy rendszerbe szervezhetjük, amiből egy automata kereskedő szoftvert lehet készíteni.

Egy 2006-os kutatásban olvastam, hogy a NASA egy genetikus algoritmus segítségével készített egy antennát, melynek a sugárzási mintája optimális a felhasználásukhoz. Ekkor gondoltam, hogy a saját felhasználásomra is tudnám kamatoztatni ezt a technikát. A genetikus algoritmusok a természeti kiválasztás folyamatain alapszanak. Arra használjuk őket, hogy optimalizációs, illetve keresési problémákhoz megbízható megoldást találjunk. Számomra az indikátorok vizsgálatánál kerülhet szóba optimalizáció, amikor el kell dönteni vásárol-e a program, vagy elad.

A szakdolgozatom fő célja, hogy egy automata kereskedő szoftver készítsek el genetikus algoritmus felhasználásával. Én azt vallom, hogy lehetséges a megfelelő hatásfok elérése, hiszen a mesterséges intelligencia már bemutatta képességeit sok területen, az én esetemben is támaszt nyújthat. Az általam célként kitűzött szimulációhoz elengedhetetlen egy adatbázis, mely a piac tényleges mozgását reprezentálja, így valóságos környezetet tudok szimulálni a stratégiám teszteléséhez. Az olvasó megismerheti a részletes taktikáját a programomnak, a technológiákat, amiket alkalmaztam, és egy átfogó képet kaphat arról, hogyan is használható fel a statisztika a részvényekkel történő kereskedés során. 